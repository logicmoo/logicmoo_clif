\documentclass[a4paper]{article}
\usepackage[utf8]{inputenc}
\usepackage[T1]{fontenc}
\usepackage{imakeidx}
\usepackage[hidelinks]{hyperref}
%% A screen friendly geometry:	    
\usepackage[paper=a5paper,scale=0.9]{geometry}
%% PPL Setup
	   
\newcommand{\assign}{\mathrel{\mathop:}=}
\newcommand{\concat}{\mathrel{+\!+}}
	    
\newcommand{\f}[1]{\mathsf{#1}}
\newcommand{\true}{\top}
\newcommand{\false}{\bot}
\newcommand{\imp}{\rightarrow}
\newcommand{\revimp}{\leftarrow}
\newcommand{\equi}{\leftrightarrow}
\newcommand{\entails}{\models}	    
\newcommand{\eqdef}{\; 
\raisebox{-0.1ex}[0mm]{$ \stackrel{\raisebox{-0.2ex}{\tiny 
\textnormal{def}}}{=} $}\; }
\newcommand{\iffdef}{\n{iff}_{\mbox{\scriptsize \textnormal{def}}}}

\newcommand{\pplmacro}[1]{\mathit{#1}}
\newcommand{\ppldefmacro}[1]{\mathit{#1}}
\newcommand{\pplparam}[1]{\mathit{#1}}
\newcommand{\pplparamidx}[2]{\mathit{#1}_{#2}}
\newcommand{\pplparamplain}[1]{#1}
\newcommand{\pplparamplainidx}[2]{#1_{#2}}
\newcommand{\pplparamsup}[2]{\mathit{#1}^{#2}}
\newcommand{\pplparamsupidx}[3]{\mathit{#1}^{#2}_{#3}}
\newcommand{\pplparamplainsup}[2]{#1^{#2}}
\newcommand{\pplparamplainsupidx}[3]{#1^{#2}_{#3}}
\newcommand{\pplparamnum}[1]{\mathit{X}_{#1}}

%%	    
%% We use @startsection just to obtain reduced vertical spacing above
%% macro headers which are immediately after other headers, e.g. of sections
%%	    
\makeatletter%
\newcounter{entry}%
\newcommand{\entrymark}[1]{}%
\newcommand\entryhead{%
\@startsection{entry}{10}{\z@}{12pt plus 2pt minus 2pt}{0pt}{}}%
\makeatother
	    
\newcommand{\pplkbBefore}
{\entryhead*{}%
\setlength{\arraycolsep}{0pt}%
\pagebreak[0]%
\begin{samepage}%
\noindent%
\rule[0.5pt]{\textwidth}{2pt}\\%
\noindent}

% \newcommand{\pplkbDefType}[1]{\hspace{\fill}{{[}#1{]}\\}}

\newcommand{\pplkbBetween}
{\setlength{\arraycolsep}{3pt}%
\\\rule[3pt]{\textwidth}{1pt}%
\par\nopagebreak\noindent Defined as\begin{center}}

\newcommand{\pplkbAfter}{\end{center}\end{samepage}\noindent}

\newcommand{\pplkbBodyBefore}{\par\noindent where\begin{center}}
\newcommand{\pplkbBodyAfter}{\end{center}}

\newcommand{\pplkbFreePredicates}[1]{\f{free\_predicates}(#1)}
% \newcommand{\pplkbRenameFreeOccurrences}[3]{\f{rename\_free\_occurrences}(#1,#2,#3)}

\newcommand{\pplIsValid}[1]{\noindent This formula is valid: $#1$\par}
\newcommand{\pplIsNotValid}[1]{\noindent This formula is not valid: $#1$\par}	    
\newcommand{\pplFailedToValidate}[1]{\noindent Failed to validate this formula: $#1$\par}

\newcounter{def}
	    
\makeindex

\begin{document}
%
% Doc at position 0
%
\title{Virtual Classes}
\date{Revision: May 10, 2016; Rendered: \today}  
\maketitle

\noindent Quine's virtual classes and related concepts.  Virtual classes are
straightforwardly expressible as macros.  Formalized with the
\href{http://cs.christophwernhard.com/pie/}{\textit{PIE}} system.
%
% Doc at position 1190
%
\section{Quine's Virtual Classes and Virtual Relations}

See
\cite[Chap.~5 The Scope of Logic]{quine:1970:philosophy-of-logic},
\cite[Chap.~I]{quine:1969:set-theory-and-its-logic-revised}.
Quine notes there that identity (that is, equality) can be ``simulated''
with subtitutivity axioms for the finite vocabulary of a given formula and
asks whether set theory can be handled analogousy.
Virtual classes and their generalization to virtual relations provide a
translation that applies to set abstraction on the right side of $\in$.

Remark: Quine cites Behmann's book \cite{beh:27} on
\cite[p.~19]{quine:1969:set-theory-and-its-logic-revised}. He notes that
Behmann also introduces operations on classes and relations as mere variant
notation for sentence connectives, as in the virtual theory of classes, but
that there is a crucial difference as Behmanns assumes classes and relations
as \emph{values of quantifiable variables}.
%
% Doc at position 2129
%
\subsection{Virtual Classes and Relations: Abstraction and Operations}

The following macros apply to both, virtual classes and relations. Relations
are like sets, except that the first argument of $\in$ and the first argument
%
% Statement at position 2395
%
\pplkbBefore
\index{inv(Y,setof(X,F_X))@$\pplparamplain{Y}\in_v\mathsf{setof}(\pplparamplain{X},\pplparamplainidx{F}{X})$}$\begin{array}{lllll}
\pplparamplain{Y}\in_v\mathsf{setof}(\pplparamplain{X},\pplparamplainidx{F}{X})
\end{array}
$\pplkbBetween
$\begin{array}{lllll}
\pplparamplainidx{F}{Y},
\end{array}
$\pplkbAfter
\pplkbBodyBefore
$
\begin{array}{l}\pplparamplainidx{F}{Y} \assign \pplparamplainidx{F}{X}[\pplparamplain{X} \mapsto \pplparamplain{Y}].

\end{array}$\pplkbBodyAfter
%
% Statement at position 2477
%
\pplkbBefore
\index{inv(X,complem(Y))@$\pplparamplain{X}\in_v\mathsf{complem}(\pplparamplain{Y})$}$\begin{array}{lllll}
\pplparamplain{X}\in_v\mathsf{complem}(\pplparamplain{Y})
\end{array}
$\pplkbBetween
$\begin{array}{lllll}
\lnot  \pplparamplain{X}\in_v\pplparamplain{Y}.
\end{array}
$\pplkbAfter
%
% Statement at position 2516
%
\pplkbBefore
\index{inv(X,isect(Y,Z))@$\pplparamplain{X}\in_v\mathsf{isect}(\pplparamplain{Y},\pplparamplain{Z})$}$\begin{array}{lllll}
\pplparamplain{X}\in_v\mathsf{isect}(\pplparamplain{Y},\pplparamplain{Z})
\end{array}
$\pplkbBetween
$\begin{array}{lllll}
\pplparamplain{X}\in_v\pplparamplain{Y} \land  \pplparamplain{X}\in_v\pplparamplain{Z}.
\end{array}
$\pplkbAfter
%
% Statement at position 2566
%
\pplkbBefore
\index{inv(X,union(Y,Z))@$\pplparamplain{X}\in_v\mathsf{union}(\pplparamplain{Y},\pplparamplain{Z})$}$\begin{array}{lllll}
\pplparamplain{X}\in_v\mathsf{union}(\pplparamplain{Y},\pplparamplain{Z})
\end{array}
$\pplkbBetween
$\begin{array}{lllll}
\pplparamplain{X}\in_v\pplparamplain{Y} \lor  \pplparamplain{X}\in_v\pplparamplain{Z}.
\end{array}
$\pplkbAfter
%
% Statement at position 2616
%
\pplkbBefore
\index{inv(X,empty)@$\pplparamplain{X}\in_v\mathsf{empty}$}$\begin{array}{lllll}
\pplparamplain{X}\in_v\mathsf{empty}
\end{array}
$\pplkbBetween
$\begin{array}{lllll}
\false .
\end{array}
$\pplkbAfter
%
% Statement at position 2645
%
\pplkbBefore
\index{inv(X,full)@$\pplparamplain{X}\in_v\mathsf{full}$}$\begin{array}{lllll}
\pplparamplain{X}\in_v\mathsf{full}
\end{array}
$\pplkbBetween
$\begin{array}{lllll}
\true .
\end{array}
$\pplkbAfter
%
% Statement at position 2672
%
\pplkbBefore
\index{inv(X,unit(Y))@$\pplparamplain{X}\in_v\mathsf{unit}(\pplparamplain{Y})$}$\begin{array}{lllll}
\pplparamplain{X}\in_v\mathsf{unit}(\pplparamplain{Y})
\end{array}
$\pplkbBetween
$\begin{array}{lllll}
\pplparamplain{X}=\pplparamplain{Y}.
\end{array}
$\pplkbAfter
%
% Statement at position 2701
%
\pplkbBefore
\index{inv(X,upair(Y,Z))@$\pplparamplain{X}\in_v\mathsf{upair}(\pplparamplain{Y},\pplparamplain{Z})$}$\begin{array}{lllll}
\pplparamplain{X}\in_v\mathsf{upair}(\pplparamplain{Y},\pplparamplain{Z})
\end{array}
$\pplkbBetween
$\begin{array}{lllll}
\pplparamplain{X}\in_v\mathsf{union}(\mathsf{unit}(\pplparamplain{Y}),\mathsf{unit}(\pplparamplain{Z})).
\end{array}
$\pplkbAfter
%
% Doc at position 2761
%
\subsection{Predicates of Virtual Classes}

The following macros apply to virtual \emph{classes}
only.
%
% Statement at position 2872
%
\pplkbBefore
\index{subseteq_v(X,Y)@$\pplparamplain{X}\subseteq_v\pplparamplain{Y}$}$\begin{array}{lllll}
\pplparamplain{X}\subseteq_v\pplparamplain{Y}
\end{array}
$\pplkbBetween
$\begin{array}{lllll}
\forall \pplparamplain{Z} \, (\pplparamplain{Z}\in_v\pplparamplain{X} \imp  \pplparamplain{Z}\in_v\pplparamplain{Y}),
\end{array}
$\pplkbAfter
\pplkbBodyBefore
$
\begin{array}{l}\pplparamplain{Z} \assign \mathrm{a\ fresh\ symbol}.

\end{array}$\pplkbBodyAfter
%
% Statement at position 2956
%
\pplkbBefore
\index{subset_v(X,Y)@$\pplparamplain{X}\subset_v\pplparamplain{Y}$}$\begin{array}{lllll}
\pplparamplain{X}\subset_v\pplparamplain{Y}
\end{array}
$\pplkbBetween
$\begin{array}{lllll}
\pplparamplain{X}\subseteq_v\pplparamplain{Y} \land  \lnot  \pplparamplain{Y}\subseteq_v\pplparamplain{X}.
\end{array}
$\pplkbAfter
%
% Statement at position 3018
%
\pplkbBefore
\index{eq_v(X,Y)@$\pplparamplain{X}=_v\pplparamplain{Y}$}$\begin{array}{lllll}
\pplparamplain{X}=_v\pplparamplain{Y}
\end{array}
$\pplkbBetween
$\begin{array}{lllll}
\forall \pplparamplain{Z} \, (\pplparamplain{Z}\in_v\pplparamplain{X} \equi  \pplparamplain{Z}\in_v\pplparamplain{Y}),
\end{array}
$\pplkbAfter
\pplkbBodyBefore
$
\begin{array}{l}\pplparamplain{Z} \assign \mathrm{a\ fresh\ symbol}.

\end{array}$\pplkbBodyAfter
%
% Doc at position 3098
%
\subsection{Predicates of Virtual Relations}

The following macros apply to virtual relations. The
relation arity has to be supplied as first argument $N$.
%
% Statement at position 3262
%
\pplkbBefore
\index{subseteq_v(N,X,Y)@$\ppldefmacro{subseteq_{v}}(\pplparamplain{N},\pplparamplain{X},\pplparamplain{Y})$}$\begin{array}{lllll}
\ppldefmacro{subseteq_{v}}(\pplparamplain{N},\pplparamplain{X},\pplparamplain{Y})
\end{array}
$\pplkbBetween
$\begin{array}{lllll}
\forall \pplparamplain{Z} \, (\pplparamplain{Z}\in_v\pplparamplain{X} \imp  \pplparamplain{Z}\in_v\pplparamplain{Y}),
\end{array}
$\pplkbAfter
\pplkbBodyBefore
$
\begin{array}{l}\pplparamplain{Z} \assign \mathrm{a\ sequence\ of\ \pplparamplain{N}\ fresh\ symbols}.

\end{array}$\pplkbBodyAfter
%
% Statement at position 3353
%
\pplkbBefore
\index{subset_v(N,X,Y)@$\ppldefmacro{subset_{v}}(\pplparamplain{N},\pplparamplain{X},\pplparamplain{Y})$}$\begin{array}{lllll}
\ppldefmacro{subset_{v}}(\pplparamplain{N},\pplparamplain{X},\pplparamplain{Y})
\end{array}
$\pplkbBetween
$\begin{array}{lllll}
\pplmacro{subseteq_{v}}(\pplparamplain{N},\pplparamplain{X},\pplparamplain{Y}) \land  \lnot  \pplmacro{subseteq_{v}}(\pplparamplain{N},\pplparamplain{Y},\pplparamplain{X}).
\end{array}
$\pplkbAfter
%
% Statement at position 3424
%
\pplkbBefore
\index{eq_v(N,X,Y)@$\ppldefmacro{eq_{v}}(\pplparamplain{N},\pplparamplain{X},\pplparamplain{Y})$}$\begin{array}{lllll}
\ppldefmacro{eq_{v}}(\pplparamplain{N},\pplparamplain{X},\pplparamplain{Y})
\end{array}
$\pplkbBetween
$\begin{array}{lllll}
\forall \pplparamplain{Z} \, (\pplparamplain{Z}\in_v\pplparamplain{X} \equi  \pplparamplain{Z}\in_v\pplparamplain{Y}),
\end{array}
$\pplkbAfter
\pplkbBodyBefore
$
\begin{array}{l}\mathrm{mac\_make\_fresh\_arg(N,Z)}.

\end{array}$\pplkbBodyAfter
%
% Doc at position 3511
%
\subsection{Operations on Binary Virtual Relations}

The macros in the following group apply to binary relations.
%
% Statement at position 3633
%
\pplkbBefore
\index{inv([X,Y],product_of_classes(A,B))@${[}\pplparamplain{X},\pplparamplain{Y}{]}\in_v\mathsf{product\_of\_classes}(\pplparamplain{A},\pplparamplain{B})$}$\begin{array}{lllll}
{[}\pplparamplain{X},\pplparamplain{Y}{]}\in_v\mathsf{product\_of\_classes}(\pplparamplain{A},\pplparamplain{B})
\end{array}
$\pplkbBetween
$\begin{array}{lllll}
\pplparamplain{X}\in_v\pplparamplain{A} \land  \pplparamplain{Y}\in_v\pplparamplain{B}.
\end{array}
$\pplkbAfter
%
% Statement at position 3697
%
\pplkbBefore
\index{inv([X,Y],converse(R))@${[}\pplparamplain{X},\pplparamplain{Y}{]}\in_v\mathsf{converse}(\pplparamplain{R})$}$\begin{array}{lllll}
{[}\pplparamplain{X},\pplparamplain{Y}{]}\in_v\mathsf{converse}(\pplparamplain{R})
\end{array}
$\pplkbBetween
$\begin{array}{lllll}
{[}\pplparamplain{Y},\pplparamplain{X}{]}\in_v\pplparamplain{R}.
\end{array}
$\pplkbAfter
%
% Statement at position 3744
%
\pplkbBefore
\index{inv([X,Z],resultant(Q,R))@${[}\pplparamplain{X},\pplparamplain{Z}{]}\in_v\mathsf{resultant}(\pplparamplain{Q},\pplparamplain{R})$}$\begin{array}{lllll}
{[}\pplparamplain{X},\pplparamplain{Z}{]}\in_v\mathsf{resultant}(\pplparamplain{Q},\pplparamplain{R})
\end{array}
$\pplkbBetween
$\begin{array}{lllll}
\exists \pplparamplain{Y} \, ({[}\pplparamplain{X},\pplparamplain{Y}{]}\in_v\pplparamplain{Q} \land  {[}\pplparamplain{Y},\pplparamplain{Z}{]}\in_v\pplparamplain{R}).
\end{array}
$\pplkbAfter
%
% Statement at position 3818
%
\pplkbBefore
\index{inv(X,image(R,A))@$\pplparamplain{X}\in_v\mathsf{image}(\pplparamplain{R},\pplparamplain{A})$}$\begin{array}{lllll}
\pplparamplain{X}\in_v\mathsf{image}(\pplparamplain{R},\pplparamplain{A})
\end{array}
$\pplkbBetween
$\begin{array}{lllll}
\exists \pplparamplain{Y} \, ({[}\pplparamplain{X},\pplparamplain{Y}{]}\in_v\pplparamplain{R} \land  \pplparamplain{Y}\in_v\pplparamplain{A}).
\end{array}
$\pplkbAfter
%
% Statement at position 3880
%
\pplkbBefore
\index{inv([X,Y],identity)@${[}\pplparamplain{X},\pplparamplain{Y}{]}\in_v\mathsf{identity}$}$\begin{array}{lllll}
{[}\pplparamplain{X},\pplparamplain{Y}{]}\in_v\mathsf{identity}
\end{array}
$\pplkbBetween
$\begin{array}{lllll}
\pplparamplain{X}=\pplparamplain{Y}.
\end{array}
$\pplkbAfter
%
% Doc at position 3917
%
\subsection{Properties of Binary Virtual Relations}


Further properties of relations can be defined as macros in terms of the
previously defined operations -- see
\cite[p.~22f]{quine:1969:set-theory-and-its-logic-revised}. For example:
%
% Statement at position 4162
%
\pplkbBefore
\index{irreflexive_v(R)@$\ppldefmacro{irreflexive_{v}}(\pplparamplain{R})$}$\begin{array}{lllll}
\ppldefmacro{irreflexive_{v}}(\pplparamplain{R})
\end{array}
$\pplkbBetween
$\begin{array}{lllll}
\pplmacro{subseteq_{v}}(\mathsf{2},\pplparamplain{R},\mathsf{complem}(\mathsf{identity})).
\end{array}
$\pplkbAfter
%
% Doc at position 4224
%
\section{Quine's Set Abstraction in Element Position}

This is discussed in \cite[p.~64ff]{quine:1970:philosophy-of-logic}.
%
% Statement at position 4356
%
\pplkbBefore
\index{inq(setof_q(X,F_X),Y)@$\mathsf{setof_{q}}(\pplparamplain{X},\pplparamplainidx{F}{X})\in_q\pplparamplain{Y}$}$\begin{array}{lllll}
\mathsf{setof_{q}}(\pplparamplain{X},\pplparamplainidx{F}{X})\in_q\pplparamplain{Y}
\end{array}
$\pplkbBetween
$\begin{array}{lllll}
\exists \pplparamplain{Z} \, (\pplparamplain{Z}\in_q\pplparamplain{Y} \land  \forall \pplparamplain{X} \, (\pplparamplain{X}\in_q\pplparamplain{Z} \equi  \pplparamplainidx{F}{X})),
\end{array}
$\pplkbAfter
\pplkbBodyBefore
$
\begin{array}{l}\pplparamplain{Z} \assign \mathrm{a\ fresh\ symbol}.

\end{array}$\pplkbBodyAfter
%
% Statement at position 4459
%
\pplkbBefore
\index{setopsq@$\ppldefmacro{setopsq}$}$\begin{array}{lllll}
\ppldefmacro{setopsq}
\end{array}
$\pplkbBetween
$\begin{array}{lllll}
\forall \mathit{y} \, \mathsf{complem_{q}}(\mathit{y})=\mathsf{setof_{q}}(\mathsf{x},\mathsf{~}(\mathsf{x}\in_q\mathit{y})) &&&&\; \land \\
\forall \mathit{y}\mathit{z} \, \mathsf{isect_{q}}(\mathit{y},\mathit{z})=\mathsf{setof_{q}}(\mathsf{x},(\mathsf{x}\in_q\mathit{y}\land\mathsf{x}\in_q\mathit{z})) &&&&\; \land \\
\forall \mathit{y}\mathit{z} \, \mathsf{union_{q}}(\mathit{y},\mathit{z})=\mathsf{setof_{q}}(\mathsf{x},(\mathsf{x}\in_q\mathit{y}\lor\mathsf{x}\in_q\mathit{z})) &&&&\; \land \\
\mathsf{empty_{q}}=\mathsf{setof_{q}}(\mathsf{x},\mathsf{false}) &&&&\; \land \\
\mathsf{full_{q}}=\mathsf{setof_{q}}(\mathsf{x},\mathsf{true}) &&&&\; \land \\
\forall \mathit{y} \, \mathsf{unit_{q}}(\mathit{y})=\mathsf{setof_{q}}(\mathsf{x},\mathsf{x}=\mathit{y}) &&&&\; \land \\
\forall \mathit{y}\mathit{z} \, \mathsf{upair_{q}}(\mathit{y},\mathit{z})=\mathsf{union_{q}}(\mathsf{unit_{q}}(\mathit{y}),\mathsf{unit_{q}}(\mathit{z})).
\end{array}
$\pplkbAfter
%
% Statement at position 4794
%
\pplkbBefore
\index{Y=setof_q(X,F_X)@$\pplparamplain{Y}=\mathsf{setof_{q}}(\pplparamplain{X},\pplparamplainidx{F}{X})$}$\begin{array}{lllll}
\pplparamplain{Y}=\mathsf{setof_{q}}(\pplparamplain{X},\pplparamplainidx{F}{X})
\end{array}
$\pplkbBetween
$\begin{array}{lllll}
\forall \pplparamplain{X} \, (\pplparamplain{X}\in_q\pplparamplain{Y} \equi  \pplparamplainidx{F}{X}).
\end{array}
$\pplkbAfter
%
% Doc at position 4846
%
\bigskip

The expansion of \textit{setopsq} is now:
\[\begin{array}{lllll}
\forall \mathit{x}\mathit{y} \, (\mathit{y}\in_q\mathsf{complem_{q}}(\mathit{x}) \equi  \lnot  \mathit{y}\in_q\mathit{x}) &&&&\; \land \\
\forall \mathit{x}\mathit{y}\mathit{z} \, (\mathit{z}\in_q\mathsf{isect_{q}}(\mathit{x},\mathit{y}) \equi  \mathit{z}\in_q\mathit{x} \land  \mathit{z}\in_q\mathit{y}) &&&&\; \land \\
\forall \mathit{x}\mathit{y}\mathit{z} \, (\mathit{z}\in_q\mathsf{union_{q}}(\mathit{x},\mathit{y}) \equi  \mathit{z}\in_q\mathit{x} \lor  \mathit{z}\in_q\mathit{y}) &&&&\; \land \\
\forall \mathit{x} \, (\mathit{x}\in_q\mathsf{empty_{q}} \equi  \false ) &&&&\; \land \\
\forall \mathit{x} \, (\mathit{x}\in_q\mathsf{full_{q}} \equi  \true ) &&&&\; \land \\
\forall \mathit{x}\mathit{y} \, (\mathit{y}\in_q\mathsf{unit_{q}}(\mathit{x}) \equi  \mathit{y}=\mathit{x}) &&&&\; \land \\
\forall \mathit{x}\mathit{y} \, \mathsf{upair_{q}}(\mathit{x},\mathit{y})=\mathsf{union_{q}}(\mathsf{unit_{q}}(\mathit{x}),\mathsf{unit_{q}}(\mathit{y})).
\end{array}
\]
%
% Doc at position 4959
%
\bibliographystyle{alpha}
\bibliography{bibscratch03}
\printindex
\end{document}
