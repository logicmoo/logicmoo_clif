\documentclass[a4paper]{article}
\usepackage[utf8]{inputenc}
\usepackage[T1]{fontenc}
\usepackage{imakeidx}
\usepackage[hidelinks]{hyperref}
%% A screen friendly geometry:	    
\usepackage[paper=a5paper,scale=0.9]{geometry}
%% PPL Setup
	   
\newcommand{\assign}{\mathrel{\mathop:}=}
\newcommand{\concat}{\mathrel{+\!+}}
	    
\newcommand{\f}[1]{\mathsf{#1}}
\newcommand{\true}{\top}
\newcommand{\false}{\bot}
\newcommand{\imp}{\rightarrow}
\newcommand{\revimp}{\leftarrow}
\newcommand{\equi}{\leftrightarrow}
\newcommand{\entails}{\models}	    
\newcommand{\eqdef}{\; 
\raisebox{-0.1ex}[0mm]{$ \stackrel{\raisebox{-0.2ex}{\tiny 
\textnormal{def}}}{=} $}\; }
\newcommand{\iffdef}{\n{iff}_{\mbox{\scriptsize \textnormal{def}}}}

\newcommand{\pplmacro}[1]{\mathit{#1}}
\newcommand{\ppldefmacro}[1]{\mathit{#1}}
\newcommand{\pplparam}[1]{\mathit{#1}}
\newcommand{\pplparamidx}[2]{\mathit{#1}_{#2}}
\newcommand{\pplparamplain}[1]{#1}
\newcommand{\pplparamplainidx}[2]{#1_{#2}}
\newcommand{\pplparamsup}[2]{\mathit{#1}^{#2}}
\newcommand{\pplparamsupidx}[3]{\mathit{#1}^{#2}_{#3}}
\newcommand{\pplparamplainsup}[2]{#1^{#2}}
\newcommand{\pplparamplainsupidx}[3]{#1^{#2}_{#3}}
\newcommand{\pplparamnum}[1]{\mathit{X}_{#1}}

%%	    
%% We use @startsection just to obtain reduced vertical spacing above
%% macro headers which are immediately after other headers, e.g. of sections
%%	    
\makeatletter%
\newcounter{entry}%
\newcommand{\entrymark}[1]{}%
\newcommand\entryhead{%
\@startsection{entry}{10}{\z@}{12pt plus 2pt minus 2pt}{0pt}{}}%
\makeatother
	    
\newcommand{\pplkbBefore}
{\entryhead*{}%
\setlength{\arraycolsep}{0pt}%
\pagebreak[0]%
\begin{samepage}%
\noindent%
\rule[0.5pt]{\textwidth}{2pt}\\%
\noindent}

% \newcommand{\pplkbDefType}[1]{\hspace{\fill}{{[}#1{]}\\}}

\newcommand{\pplkbBetween}
{\setlength{\arraycolsep}{3pt}%
\\\rule[3pt]{\textwidth}{1pt}%
\par\nopagebreak\noindent Defined as\begin{center}}

\newcommand{\pplkbAfter}{\end{center}\end{samepage}\noindent}

\newcommand{\pplkbBodyBefore}{\par\noindent where\begin{center}}
\newcommand{\pplkbBodyAfter}{\end{center}}

\newcommand{\pplkbFreePredicates}[1]{\f{free\_predicates}(#1)}
% \newcommand{\pplkbRenameFreeOccurrences}[3]{\f{rename\_free\_occurrences}(#1,#2,#3)}

\newcommand{\pplIsValid}[1]{\noindent This formula is valid: $#1$\par}
\newcommand{\pplIsNotValid}[1]{\noindent This formula is not valid: $#1$\par}	    
\newcommand{\pplFailedToValidate}[1]{\noindent Failed to validate this formula: $#1$\par}

\newcounter{def}
	    
\makeindex

\begin{document}
%
% Doc at position 0
%
\title{Circumscription}
\date{Revision: February 13, 2018; Rendered: \today}
\maketitle

\noindent Definition of predicate circumscription.  Formalized with the
\href{http://cs.christophwernhard.com/pie/}{\textit{PIE}} system.

%
% Doc at position 278
%
\section{Definition of Predicate Circumscription}

%
% Statement at position 337
%
\pplkbBefore
\index{xcirc(S,F)@$\ppldefmacro{xcirc}(\pplparamplain{S},\pplparamplain{F})$}$\begin{array}{lllll}
\ppldefmacro{xcirc}(\pplparamplain{S},\pplparamplain{F})
\end{array}
$\pplkbBetween
$\begin{array}{lllll}
\pplparamplain{F} \land  \lnot  \pplmacro{xraise}(\pplparamplain{S},\pplparamplain{F}).
\end{array}
$\pplkbAfter
%
% Doc at position 377
%
A version of parallel predicate circumscription
\cite{lifschitz:circumscription:94}.  The $F$ parameter is the circumscribed
formula.  The $S$ parameter specifies the roles of the predicates in the
circumscription.  It is a list of specifiers of the following form, where $p$
is a predicate and the second component indicates positive, negative, or both
polarities:

\begin{center}
\begin{tabular}{ll}
$p$ is to be minimized & $p \mathsf{\textrm{-}n}$\\
$p$ is to be maximized & $p \mathsf{\textrm{-}p}$\\
$p$ is varying & $p\mathsf{\textrm{-}pn}$\\
$p$ is fixed & $p$ is not mentioned in $S$
\end{tabular}
\end{center}
%
In some cases it might be necessary to specify explicitly also the arity of
the respective predicates, e.g. $p \mathsf{/1\textrm{-}n}$.

The $S$ argument is considered complementary to the scope argument to
$\pplmacro{circ}$ in \cite{cw-projcirc} (like in forgetting instead of
projection).

%
% Statement at position 1300
%
\pplkbBefore
\index{xraise(S,F)@$\ppldefmacro{xraise}(\pplparamplain{S},\pplparamplain{F})$}$\begin{array}{lllll}
\ppldefmacro{xraise}(\pplparamplain{S},\pplparamplain{F})
\end{array}
$\pplkbBetween
$\begin{array}{lllll}
\exists \pplparamplain{Q} \, (\pplparamplainidx{F}{1} \land  \pplparamplainidx{T}{1} \land  \lnot  \pplparamplainidx{T}{2}),
\end{array}
$\pplkbAfter
\pplkbBodyBefore
$
\begin{array}{l}\pplparamplainidx{F}{1} \assign \pplparamplain{F}[\pplparamplain{S} \mapsto \pplparamplain{Q}],\\
\pplparamplainidx{S}{2} \assign \pplparamplain{S}\; \mathrm{with\ arities\ from}\; \pplparamplain{F},\\
\pplparamplainidx{T}{1} \assign \mathrm{transfer\ clauses}\; \pplparamplainidx{S}{2} \rightarrow \pplparamplain{Q},\\
\pplparamplainidx{T}{2} \assign \mathrm{transfer\ clauses}\; \pplparamplain{Q} \rightarrow \pplparamplainidx{S}{2}.

\end{array}$\pplkbBodyAfter
%
% Doc at position 1491
%
The second-order subformula on which $\pplmacro{xcirc}$ is based, Similar
to $\pplmacro{raise}$ \cite{cw-projcirc}, however the scope argument $S$ is
considered complementary (like in forgetting instead of projection).

%
% Doc at position 1799
%
\section{Some Examples}

These are examples from \cite{lifschitz:circumscription:94}.

\bigskip


\noindent Input: $\pplmacro{xcirc}({[}\mathsf{p}\textrm{-}\mathsf{n}{]},\mathsf{p}\mathsf{a}).$\\
\noindent Result of elimination:
\[\begin{array}{lllll}
\mathsf{p}\mathsf{a} \land  \forall \mathit{x} \, (\mathsf{p}\mathit{x} \imp  \mathit{x}=\mathsf{a}).
\end{array}
\]

\noindent Input: $\pplmacro{xcirc}({[}\mathsf{p}\textrm{-}\mathsf{n}{]},\lnot  \mathsf{p}\mathsf{a}).$\\
\noindent Result of elimination:
\[\begin{array}{lllll}
\forall \mathit{x} \, \lnot  \mathsf{p}\mathit{x}.
\end{array}
\]

\noindent Input: $\pplmacro{xcirc}({[}\mathsf{p}\textrm{-}\mathsf{n}{]},(\mathsf{p}\mathsf{a} \land  \mathsf{p}\mathsf{b})).$\\
\noindent Result of elimination:
\[\begin{array}{lllll}
\mathsf{p}\mathsf{a} \land  \mathsf{p}\mathsf{b} \land  \forall \mathit{x} \, (\mathsf{p}\mathit{x} \imp  \mathit{x}=\mathsf{a} \lor  \mathit{x}=\mathsf{b}).
\end{array}
\]

\noindent Input: $\pplmacro{xcirc}({[}\mathsf{p}\textrm{-}\mathsf{n}{]},(\mathsf{p}\mathsf{a} \lor  \mathsf{p}\mathsf{b})).$\\
\noindent Result of elimination:
\[\begin{array}{lllll}
(\mathsf{p}\mathsf{a} \lor  \mathsf{p}\mathsf{b}) \land  \forall \mathit{x} \, (\mathsf{p}\mathit{x} \imp  (\mathsf{p}\mathsf{a} \imp  \mathit{x}=\mathsf{a}) \land  (\mathsf{p}\mathsf{b} \imp  \mathit{x}=\mathsf{b})).
\end{array}
\]
\pplIsValid{\pplmacro{last\_result} \equi  \forall \mathit{x} \, (\mathsf{p}\mathit{x} \equi  \mathit{x}=\mathsf{a}) \lor  \forall \mathit{x} \, (\mathsf{p}\mathit{x} \equi  \mathit{x}=\mathsf{b}).}
%
% Doc at position 2253
%

\medskip


\noindent Input: $\pplmacro{xcirc}({[}\mathsf{p}\textrm{-}\mathsf{n}{]},(\lnot  \mathsf{p}\mathsf{a} \lor  \mathsf{p}\mathsf{b})).$\\
\noindent Result of elimination:
\[\begin{array}{lllll}
\forall \mathit{x} \, \lnot  \mathsf{p}\mathit{x}.
\end{array}
\]

\noindent Input: $\pplmacro{xcirc}({[}\mathsf{p}\textrm{-}\mathsf{n}{]},(\mathsf{p}\mathsf{a} \lor  (\mathsf{p}\mathsf{b} \land  \mathsf{p}\mathsf{c}))).$\\
\noindent Result of elimination:
\[\begin{array}{lllll}
(\mathsf{p}\mathsf{a} \lor  (\mathsf{p}\mathsf{b} \land  \mathsf{p}\mathsf{c})) &&&&\; \land \\
\forall \mathit{x} \, (\mathsf{p}\mathit{x} \imp  (\mathsf{p}\mathsf{a} \imp  \mathit{x}=\mathsf{a}) \land  (\mathsf{p}\mathsf{b} \land  \mathsf{p}\mathsf{c} \imp  \mathit{x}=\mathsf{b} \lor  \mathit{x}=\mathsf{c})).
\end{array}
\]
\pplIsValid{\pplmacro{last\_result} \equi  \forall \mathit{x} \, (\mathsf{p}\mathit{x} \equi  \mathit{x}=\mathsf{a}) \lor  (\forall \mathit{x} \, (\mathsf{p}\mathit{x} \equi  \mathit{x}=\mathsf{b} \lor  \mathit{x}=\mathsf{c}) \land  \mathsf{a}\neq \mathsf{b} \land  \mathsf{a}\neq \mathsf{c}).}
%
% Doc at position 2579
%

\medskip


\noindent Input: $\pplmacro{xcirc}({[}\mathsf{p}\textrm{-}\mathsf{n}{]},\forall \mathit{x} \, \mathsf{p}\mathit{x}).$\\
\noindent Result of elimination:
\[\begin{array}{lllll}
\forall \mathit{x} \, \mathsf{p}\mathit{x}.
\end{array}
\]

\noindent Input: $\pplmacro{xcirc}({[}\mathsf{p}\textrm{-}\mathsf{n}{]},\forall \mathit{x} \, (\mathsf{q}\mathit{x} \imp  \mathsf{p}\mathit{x})).$\\
\noindent Result of elimination:
\[\begin{array}{lllll}
(\forall \mathit{x} \, (\mathsf{q}\mathit{x} \imp  \mathsf{p}\mathit{x}) \imp  \forall \mathit{x} \, (\mathsf{p}\mathit{x} \imp  \mathsf{q}\mathit{x})) &&&&\; \land \\
\forall \mathit{x} \, (\mathsf{q}\mathit{x} \imp  \mathsf{p}\mathit{x}).
\end{array}
\]

\noindent Input: $\pplmacro{xcirc}({[}\mathsf{p}\textrm{-}\mathsf{n}{]},\exists \mathit{x} \, \mathsf{p}\mathit{x}).$\\
\noindent Result of elimination:
\[\begin{array}{lllll}
\forall \mathit{x} \, (\mathsf{p}\mathit{x} \imp  \forall \mathit{y} \, (\mathsf{p}\mathit{y} \imp  \mathit{x}=\mathit{y})) \land  \exists \mathit{x} \, \mathsf{p}\mathit{x}.
\end{array}
\]
\pplIsValid{\pplmacro{last\_result} \equi  \exists \mathit{x} \, \forall \mathit{y} \, (\mathsf{p}\mathit{y} \equi  \mathit{x}=\mathit{y}).}
%
% Doc at position 2861
%

\medskip


\noindent Input: $\pplmacro{xcirc}({[}\mathsf{p}\textrm{-}\mathsf{n}{]},\forall \mathit{x} \, \mathsf{p}\mathit{x}\mathit{x}).$\\
\noindent Result of elimination:
\[\begin{array}{lllll}
(\forall \mathit{x}\mathit{y} \, (\mathsf{p}\mathit{x}\mathit{y} \imp  \mathit{x}=\mathit{y}) \lor  \exists \mathit{x} \, \lnot  \mathsf{p}\mathit{x}\mathit{x}) &&&&\; \land \\
\forall \mathit{x} \, \mathsf{p}\mathit{x}\mathit{x}.
\end{array}
\]

\noindent Input: $\pplmacro{xcirc}({[}\mathsf{p}\textrm{-}\mathsf{n},\mathsf{q}\textrm{-}\mathsf{pn}{]},\forall \mathit{x} \, (\mathsf{q}\mathit{x} \imp  \mathsf{p}\mathit{x})).$\\
\noindent Result of elimination:
\[\begin{array}{lllll}
\forall \mathit{x} \, (\mathsf{q}\mathit{x} \imp  \mathsf{p}\mathit{x}) \land  \forall \mathit{x} \, \lnot  \mathsf{p}\mathit{x}.
\end{array}
\]
\pplIsValid{\pplmacro{last\_result} \equi  \forall \mathit{x} \, \lnot  \mathsf{p}\mathit{x} \land  \forall \mathit{x} \, \lnot  \mathsf{q}\mathit{x}.}
%
% Doc at position 3317
%

\medskip

%
% Statement at position 3336
%
\pplkbBefore
\index{block_axioms@$\ppldefmacro{block\_axioms}$}$\begin{array}{lllll}
\ppldefmacro{block\_axioms}
\end{array}
$\pplkbBetween
$\begin{array}{lllll}
\forall \mathit{x} \, (\mathsf{block}(\mathit{x}) \land  \lnot  \mathsf{ab}(\mathit{x}) \imp  \mathsf{ontable}(\mathit{x})) &&&&\; \land \\
\lnot  \mathsf{ontable}(\mathsf{b_{1}}) &&&&\; \land \\
\mathsf{block}(\mathsf{b_{1}}) &&&&\; \land \\
\mathsf{block}(\mathsf{b_{2}}) &&&&\; \land \\
\mathsf{b_{1}}\neq \mathsf{b_{2}}.
\end{array}
$\pplkbAfter

\noindent Input: $\pplmacro{xcirc}({[}\mathsf{ab}\textrm{-}\mathsf{n},\mathsf{ontable}\textrm{-}\mathsf{pn}{]},\pplmacro{block\_axioms}).$\\
\noindent Result of elimination:
\[\begin{array}{lllll}
\mathsf{b_{1}}\neq \mathsf{b_{2}} &&&&\; \land \\
\lnot  \mathsf{ontable}(\mathsf{b_{1}}) &&&&\; \land \\
\mathsf{block}(\mathsf{b_{1}}) &&&&\; \land \\
\mathsf{block}(\mathsf{b_{2}}) &&&&\; \land \\
(\mathsf{ab}(\mathsf{b_{1}}) \imp  \forall \mathit{x} \, (\mathsf{ab}(\mathit{x}) \imp  \mathit{x}=\mathsf{b_{1}} \land  \mathsf{block}(\mathit{x}))) &&&&\; \land \\
\forall \mathit{x} \, (\mathsf{block}(\mathit{x}) \imp  \mathsf{ab}(\mathit{x}) \lor  \mathsf{ontable}(\mathit{x})).
\end{array}
\]
\pplIsValid{\pplmacro{last\_result} \equi  \pplmacro{block\_axioms} \land  \forall \mathit{x} \, (\mathsf{ab}(\mathit{x}) \equi  \mathit{x}=\mathsf{b_{1}}).}
%
% Doc at position 3620
%

\subsection{Auxiliary Macros}

%
% Statement at position 3659
%
\pplkbBefore
\index{last_result@$\ppldefmacro{last\_result}$}$\begin{array}{lllll}
\ppldefmacro{last\_result}
\end{array}
$\pplkbBetween
$\begin{array}{lllll}
\pplparamplain{X},
\end{array}
$\pplkbAfter
\pplkbBodyBefore
$
\begin{array}{l}\mathrm{last\_ppl\_result(X)}.

\end{array}$\pplkbBodyAfter
%
% Doc at position 3707
%
\bibliographystyle{alpha}
\bibliography{bibscratch03}
\printindex
\end{document}
